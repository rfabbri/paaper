%\documentclass[letterpaper,peerreview,12pt,compsoc,draftcls]{IEEEtran}
%\documentclass[10pt,twocolumn,letterpaper]{article}
\documentclass[letterpaper]{article}

%
% Final Checklist TODO XXX 
% - Spellcheck
% - check xxx's todo's etc
%
%\usepackage{ruler}
%\usepackage{cvpr}
%\usepackage{iccv}
\usepackage{times}
\usepackage{epsfig}
\usepackage{graphicx}
\usepackage{amsmath}
\usepackage{amssymb}
%\usepackage{amsthm}
\usepackage{mathtools,empheq}
\usepackage[tight,footnotesize]{subfigure}
%\usepackage{caption}
\usepackage{verbatim}
\usepackage{color}
\usepackage{nomencl}
\usepackage[nocompress]{cite}
\usepackage[pagebackref=true,breaklinks=true,letterpaper=true,colorlinks,bookmarks=false]{hyperref}
\usepackage{algorithm}
\usepackage{enumerate}
\usepackage{multirow}
\usepackage[width=122mm,left=12mm,paperwidth=146mm,height=193mm,top=12mm,paperheight=217mm]{geometry}


\newcommand{\intinf}{\int_{-\infty}^{\infty}}
\newcommand{\argmin}{\operatornamewithlimits{argmin}}

\newcommand{\grad}{\nabla}
\newcommand{\slant}{\sigma}
\newcommand{\R}{\mathbb{R}} % the reals
\newcommand{\skewm}[1]{{#1}_\times}

\renewcommand{\vec}[1]{\mathbf{#1}}

\newcommand{\lbar}{\overline}

\newcommand{\id}{\text{\emph{I}}}
\newcommand{\ransac}{\textsc{ransac}}
\newcommand{\sift}{\textsc{sift}}
\newcommand{\klt}{\textsc{klt}}
\newcommand{\svd}{\textsc{svd}}
\newcommand{\sfm}{\textsc{sfm}}


\newtheorem{thm}{Theorem}
\newtheorem{lem}[thm]{Lemma}

%\newtheorem{theorem}{Theorem}[section]
%\newtheorem{corollary}[theorem]{Corollary}
\newtheorem{corolary}[theorem]{Corollary}
%\newtheorem{proposition}[theorem]{Proposition}
%\newtheorem{lemma}[theorem]{Lemma}

%\theoremstyle{definition}
%\newtheorem{definition}{Definition}
%\newtheorem{remark}{Remark}[section]
\newtheorem{assumption}{Assumption}[section]
%\newtheorem{problem}{Problem}[section]
%\newtheorem{question}{Question}[section]
%\newtheorem{property}{Property}
\newtheorem{transformation}{Transformation}

\numberwithin{equation}{section}

\newcommand{\Gama}{\boldsymbol{\Gamma}}
\newcommand{\gama}{\boldsymbol{\gamma}}
\newcommand{\bsigma}{\boldsymbol{\sigma}}
%\newcommand{\Gama}{\Gamma}
%\newcommand{\gama}{\gamma}
\newcommand{\T}{\boldsymbol{T}}
\newcommand{\N}{\mathbf{N}}
\newcommand{\NSurface}{\mathbf{N}}
\newcommand{\Nlocal}{\overline{\N}} % normal in local coordinates
\newcommand{\balpha}{\boldsymbol{\alpha}}
\newcommand{\tDt}{t+\Delta t}
\newcommand{\bpsi}{\boldsymbol{\boldsymbol{\psi}}}
\newcommand{\bp}{\mathbf p}
\newcommand{\deldt}[1]{\frac{\partial#1}{\partial t}}
\newcommand{\ddt}[1]{\frac{d #1}{dt}}
\newcommand{\delds}[1]{\frac{\partial#1}{\partial s}}
\newcommand{\mybar}[1]{\overline{#1}}
\newcommand{\norm}[1]{\|#1\|}
\newcommand{\I}{\mathbf{I}}
\newcommand{\brho}{\boldsymbol{\rho}}
\newcommand{\lightrgb}{\boldsymbol{l}}
\newcommand{\B}{\boldsymbol{B}}
\renewcommand{\t}{\boldsymbol{t}}
\newcommand{\n}{\boldsymbol{n}}
\renewcommand{\b}{\boldsymbol{b}}
\newcommand{\e}{\boldsymbol{e}}
\newcommand{\f}{\boldsymbol{e}_3}
\newcommand{\ff}{\mathbf{f}}
\newcommand{\hf}{\boldsymbol{\hat{f}}}
\newcommand{\g}{\boldsymbol{g}}
\newcommand{\G}{\boldsymbol{G}}
\newcommand{\bc}{\boldsymbol{c}}
\newcommand{\Curve}{\Gamma}
%\newcommand{\X}{\boldsymbol{X}}
%\newcommand{\x}{\boldsymbol{x}}
\newcommand{\X}{\mathbf{X}}
\newcommand{\x}{\mathbf{x}}
\newcommand{\tilx}{\tilde x}
\newcommand{\tily}{\tilde y}
\newcommand{\tilgama}{\tilde \gama}
\newcommand{\ugama}{\hat{\gama}} %unit gama
\newcommand{\br}{\bar r}
\newcommand{\Kc}{\mathbf K_c}
\newcommand{\Kim} {\mathcal K_{im}}
\newcommand{\lepi}{\mathbf r}
\newcommand{\itan}{\tan^{-1}}
\newcommand{\uu}{\xi}
\newcommand{\buu}{\bar \uu}
\newcommand{\bvv}{\bar \vv}
\newcommand{\vv}{\eta}
\newcommand{\VV}{\mathbf{V}} % translational velocity
\newcommand{\VVspeed}{V} % translational velocity
\newcommand{\field}{\boldsymbol\chi}
\newcommand{\ufield}{\hat{\boldsymbol{\chi}}}
\newcommand{\fieldc}{\chi} % field component
\newcommand{\transl}{\mathcal{T}}
\newcommand{\rot}{\mathcal{R}}
\newcommand{\albedo}{\alpha}
\newcommand{\depth}{\rho}      % depth as z
\newcommand{\udepth}{{\hat{\rho}}} % depth along ray
\newcommand{\ttransl}{\T} % translation tangent
\newcommand{\surface}{\mathcal{M}} % surface/manifold
\newcommand{\surf}{\mathcal{M}} % surface/manifold short
\newcommand{\jacm}{\mathtt{J}} % Jacobian matrix
\newcommand{\xbar}{\bar x}
\newcommand{\ybar}{\bar y}
\newcommand{\zbar}{\bar z}

\newcommand{\bdelta}{\boldsymbol \delta}
%\newcommand{\X}{\boldsymbol{X}}
%\newcommand{\x}{\boldsymbol{x}}
%\newcommand{\X}{\mathbf{X}}
%\newcommand{\x}{\mathbf{x}}
\newcommand{\boldu}{\mathbf{u}}
\newcommand{\boldv}{\mathbf{v}}
\newcommand{\boldw}{\mathbf{w}}
% The following are not very good constructs it seems. Better to use just
% \begin{bmatrix}..\end{\bmatrix}
\newcommand{\datsqbr}[2][rrrrrrrrrrrrrrrrrrrrrrrrrrrrrrrrrrrr]{\left[
\begin{array}{#1}
#2\\
\end{array}
\right]
}
\newcommand{\tgtveloc}{\tilde\alpha} % real tangential velocity
\newcommand{\trace}{\text{trace}\,}
\newcommand{\benx}{x} % ben's aux. variable in da calibration paper

\newcommand{\ie}{{\it i.e.}}
\newcommand{\etc}{{\it etc}}
\newcommand{\eg}{{\it e.g.}}
\newcommand{\wrt}{{\it w.r.t. }}
\newcommand{\etal}{{\it et.~al.}}


\usepackage{url}
\usepackage{graphicx}
\usepackage[tight,footnotesize]{subfigure}
\usepackage{color}
\usepackage{verbatim}
\usepackage{multirow}
\usepackage[pagebackref=true,breaklinks=true,letterpaper=true,colorlinks,bookmarks=false]{hyperref}

%%%%%%%%%%%%%%%%%%%%%%%%%%%%%%%%%%%%%%%%
% You have two versions of the macro
% \draftnote{My note}. The first version puts notes (e.g. My note in the example)
% into the margin of your document. The second formats the note as nothing. You
% 'comment out' the version of the macro you don't want (using a % at the
% beginning of the line).
%\newcommand{\draftnote}[1]{\marginpar{\tiny\raggedright\textsf{\hspace{0pt}#1}}}
\newcommand{\draftnote}[1]{\marginpar{\tiny\raggedright\textsf{\hspace{0pt}#1}}}
%\newcommand{\draftnote}[1]{}

% This one is just for the comments for in-line text.
\newcommand{\indraftnote}[1]{\textcolor{blue}{\texttt{\footnotesize[#1]}}}
\newcommand{\todo}[1]{\indraftnote{todo: #1}}
%\newcommand{\indraftnote}[1]{}


\begin{document}
\title{
    AA: the Algoritmic Autoregulation (Software Development) Methodology
}

\author{%
Renato~Fabbri \and Ricardo~Fabbri \and Vilson Vieira \and Alexandre Negrao \and Lucas Zambianchi
\and Marcos Mendonca \and Danilo Shiga
}

\maketitle
%\thispagestyle{empty}

\begin{abstract}
We present a new methodology for coordinating
teamwork called Algorithmic Autoregulation (AA). This methodology is based on recent social networking concepts and individual merit and exhibits convenient asynchronous, autodocumenting and autoregulating characteristics. Team
members take on an egalitarian and volutary role, by logging into
periodic ``AA sessions'' for an arbitrary duration (e.g.\ 2 hours per day).
During each session, a user creates a log composed of short text sentences
about their activity. These logs are
publicly aggregated in a Website and are peer-validated, as in code
review. A short screencast is ideally recorded at the end of each session to make
AA logs more understandable. This methodology seems to be well-suited for
increasing the efficiency of teams working on
Global Software Development (GSD). Recognized reasons for this are
the: 1) builtin asynchronous
on-demand communication, documentation and working hours comprobatives;
2) reduced need for central management,
meetings and time-consuming reports. Hence, AA
legitimizes and eases activities of a distributed software team.  It
enables groups to have new means to fund activities,
allowing for business models to emerge from
distributed software development. AA is proposed,
at it's core, as a way of having self-replicating hacker initiatives. These claims are
discussed in a real case-study of a distributed free software
hacker team called Lab Macambira.
\end{abstract}



\section{Introduction}

One of the defining features of modern times is the widening geographical
distribution of software teams~\cite{last2003}. This is responsible for the so
called Global Software Development (GSD)~\cite{german2003}.  An example is the
free software movement. Projects and institutions like Mozilla Foundation has
several employees and thousands of voluntary developers distributed across many
countries. The same is true for GNOME~\cite{german2003}, OpenBSD, MySQL or
Apache Software Foundation, to cite just a few\footnote{Ohloh, the open source
  network, have a more complete and constantly updated list of the most active
  projects on-line at \url{www.ohloh.net}}. Along the free and open software
projects, GSD has a growing popularity in every niche of the software industry
as a whole, even on those distributing their software with proprietary
licenses. This phenomenon is attributed to a variety of factors such as a larger
labor pool, natural globalization of software companies and foundations or even
the premise of cheaper cost of production~\cite{komi2005}.

Despite the advantages of GSD, it is known how difficult it is to coordinate and
fund free software on a larger scale than currently available. That is, when
teams are very heterogeneous, containing not only volunteers and experienced
developers, but also contractors from different backgrounds and cultures. This
observations are in consonance with dedicated studies, such
as~\cite{carmel1999}.  As a common base, related difficulties for GSD gather
around distance, time and cultural differences. In the case of free or open
software projects, all these factors are involved.

Another problem faced by modern software companies and other collectives are
frequent ineffective meetings, which are seldom focused on the interest of any
attendant. The result is that it has become the norm to participate in these
with the ``laptop open'', which is unproductive. Software developers like to
code, to be productive, to have their hands on their project, to do what they
are best at. They dislike to have to stop for meetings or to write lengthy
reports to justify their funding.
%\todo{ler mais. cacm, etc}

To address these matters is the purpose of AA, a methodology and an associated
software system for coordinating distributed team work dealing with the
disadvantages of GSD. Team members take on an egalitarian role, and stay
voluntarily \textit{logged} in the system for part of their time (e.g.\ 2 hours
per day), during which they log a periodical short text sentence --- similar to
a ``tweet'' in this Twitter era --- as the status of their activity using an
easy to use command. These ``micro-blog sentences'' are publicly aggregated and
validated by other team members. Through AA, the community has a methodology and
an associated system to validate and enable the activities of a distributed
software team. It implicitly legitimizes financial support for the expansion of
the activity of the developer team as the activities are documented and
validated.  The AA methodology is specially useful for coordinating distributed
and decentralized team work, providing effective means to asynchronously update
different team members, diminishing the need for harsh synchronous meetings.

%\todo{alerts work for greater consciousness of time}
%\todo{Integrate notes from Ricardo+Renato april 15/16 meeting}
%\todo{integrate IRC chat notes}

A brief overview of current papers about GSD methodologies related with AA is
presented in section~\ref{related-work}. In section~\ref{aa-methodology}, the
most relevant characteristics of the AA methodology are outlined. In this same
section, there is a description of an AA use in a team of 9 developers in the
second half of 2011 and a broader a more released use of AA from 2012 until
now. Finally, in Section~\ref{conclusions}, there are final conclusions and
indicatives of future possibilities to the practical use of AA on other teams of
software developers or individuals working on non-software activities.

\indraftnote{
A very good article on the value of asynchronous communication for personal
and group productivity, related to the key necessity of having moments of
introversion to avoid daily pressures of forced socialization. The way we work
on the digital age enables people to be very productive, the article also
mentions Linux as a hallmark example~\cite{Thompson:Wired:2012}
}

\indraftnote{TODO: cite CIA.vc bot stuff}

\section{Related Work}
\label{related-work}

%% aqui tomei como ponto de vista as metodologias associadas a GSD
%% (Global Software Development) que considero foco do AA, e sua
%% principal vantagem

%\todo{survey other methodologies such as agile etc}
%\todo{probably a good source http://agilemanifesto.org/}

There has been a large amount of research done in the area of methodologies to
deal with distributed teams of developers. We are focusing in GSD here, however
some principles involved on those methodologies could be used on smaller teams
of developers working in the same place, time and with minor cultural
differences. Moreover we generally think on ``distributed development'' being
global which is not totally true. We even applied AA to a team that work at the
same city but on different times (more details on Section~\ref{results}). Even
smaller groups of developers working on the same building could use GSD
methodologies. A thorough survey of these methodologies is beyond the scope of
this paper. Here we present a brief overview.

Various methodologies for GSD were built around the factors that affect
distributed team works, proposed by Carmel~\cite{carmel1999} and comprising
three distances: geographical, cultural and temporal. Geographical distance
prejudice \emph{coordination}, being the act of integrating all the tasks
distributed between units~\cite{carmel2001}; \emph{control}, or the process to
maintain specific goals, policies or quality levels; and
\emph{communication}. All these factors are correlated, for example, a team need
to have clearly communication to work on tasks of a specific problem. Cultural
distance encompass differences on organizational and natural culture. Spoken
language, unit and ethnic values are common forms of this distance. Some
companies prefer to situate development units in foreign locations with minimal
cultural distance (e.g.\ an American firm prefer Ireland, because of spoken
language similarity~\cite{carmel2001}). And finally the temporal distance that
prejudices synchronous communications like telephone or video-conferences. Units
of developers working on different time-zones are concerned with managing of
their agenda guided by this temporal distance.

Targeting geographical distance, Carmel~\cite{carmel2001} suggests a tactic to
reduce intensive collaboration. His approach divides the whole software
life-cycle on levels of complexity. Each level has a degree of
collaboration. For example, some developers working on a project with high
collaboration level should use the follow-the-sun approach: when concluding the
work day, they pass their work to the team working on another time-zone. Other
tactics are suggested by the same author to deal with the three distances, like
separating foreign units of developers in time-zone bands.

Battin et al.~\cite{battin2001} propose and argues about their experiments using
specific methodologies created for the distributed development centers of the
Motorola Company (which has over 25 software development centers
worldwide). These methodologies included constant communication with critical
units, incremental integration and scheduled based on time-zones distributed to
developers on 6 countries from 3 continents.

While considering free software projects instead of companies, the same factors
are present and some methodologies arises. German~\cite{german2003} gives a
concise review of methodologies used by the GNOME project, one of the most
active free software projects and used by companies like Sun Microsystems. It is
interesting to note a difference on viewpoint present on his paper: German
focuses the methodology description on code. He start explaining that GNOME is
separated into modules (76 on version 2.4, to be precise) and each module has
one maintainer who divide his modules into separated parts in which other
developers can work on independent tasks, along other responsibilities. As like
others free software projects, all the development was done using a
\emph{bugtracker} to bugs and issues management, mail lists and Internet Relay
Chat (IRC) to discussion and communication and a software configuration
management like SVN or Git. Periodical (commonly yearly) conferences like GUADEC
is common on free and open source projects to face-to-face meeting and is based
each time on a different place.

\section{The AA Methodology}
\label{aa-methodology}

As noted, some strategies for GSD is based on complex methodologies and many of
those were built for a specific company or software center. Here we propose an
alternative methodology based on a simple idea: small working sessions logged by
a computational tool. Figure~\ref{fig:mm} summarizes our methodology.

\begin{figure}
\begin{center}
   \includegraphics[width=0.8\linewidth,keepaspectratio=true]{figs/aa-mm.png}
\end{center}
   \caption{
   Mindmap of our methodology
   }
\label{fig:mm}
\end{figure}

\subsection{AA Session}

From the developer perspective, the AA methodology is based on creating pretty
small high perspective reports of what they are doing in a specific time frame,
that can be something between 5 to 15 minutes, depending in what is more
comfortable for the developer. An AA Session would be a period of at least 2
hours doing these reports. The developer can set reminders to show up when its
time to make a new report. The objective of the flexible time frame and
reminders is to minimize developer overhead during his AA session. In this way
he can make the reports while staying concentrated in his code. Each report can
be sent directly to an on-line server, or stored locally in a temporarily
database for sending later. This make possible the use of AA while offline.

Developers can also record a video screencast in the end of the session
summarizing what has been done, explaining with his words and showing his most
important results. This, combined with the textual log of his session, makes the
whole report more understandable for himself or other people searching for
information about his production.

\subsection{AA Website Report}

\begin{figure}
\begin{center}
   \includegraphics[width=0.95\linewidth]{figs/aa-0_1.png}
\end{center}
   \caption{AA Version 0.1}
\label{fig:aaserver}
\end{figure}

All AA reports made by the developers are sent to a Web server and become public
in a Website, being possible then for a manager or another developer to follow
very closely the work of a developer, nearly real-time, reading each of his
small reports of what he is doing right now.

Another possibility is to check older sessions to see when sometimes was done
and the comments of the developer about it, since each AA post happens in a very
small time frame of work, the information about what was done become very easy
to understand, instead of a long report in the end of a session.

The site is where the developer can add to his session a link for his screencast
about that session, a video summary of what was done to complement the reports,
useful specially on cases where the small reports were done in a hurry, because
the developer did not want to lose his focus on something important at that
moment.

\subsection{Validation}

Each AA session must be validated by another developer, it means that all
reports are read by someone that will consider then valid or not and will even
write commentaries about the specific session. The developer in charge of
validating a session is decided randomly by the AA Web server, which send an
email to the chosen developer with an URL to a validation interface.

% TODO: adicionar figura aa cliente (ou tabela com os principais comandos)
% TODO: adicionar figura com arquitetura do AA cliente + AA web + validação
% TODO: melhorar screenshot do AA web

\section{Results and Discussion}
\label{results}

The easy and effective management of teams working on GSD is the main purpose of
the AA methodology. We applied this methodology to a group of 9 developers in
July of 2011 on what we called Lab Macambira~\footnote{LabMacambira.sf.net:
  \url{http://labmacambira.sf.net}}. The main objective of the team was to work
on different free software projects, contributing directly to their development,
sending bug corrections or proposing new features on their source code.

All the team members had different levels of knowledge on software development
as part of large and distributed free software projects like Scilab or
Mozilla. In this way, one month of training was conducted by two experienced
developers, teaching the basics of use of development support tools like
bugtrackers, programming languages and version control systems. After this
period, a challenge was proposed for the new developers: send a bug correction
or a new feature to a large free software project and you will work with us, you
will be a ``Macambira'' developer. Table~\ref{tabela:contribuicoes} summarizes
the contributions of each ``Macambira'' to free software projects.

%% adicionar tabela de contribuições

\begin{table}
  \caption{Free and open software projects that received contributions
    from ``Macambiras''. On the first column we can see a list of
    applications of those projects. At right, the pseudonym of the
    ``Macambiras'' who sent \textit{commits} to the application. At
    ``Lab Macambira'', and at free software community in general, is a
    common practice to use pseudonym as identification.}
    \small\begin{tabular}{|l|l|}
        \hline
        Application           & ``Commiters''                       \\
        \hline \hline
        Mozilla Firefox       & daneoshiga, bzum                    \\
        Evince                & hick209, bzum, marcicano, mquasar   \\
        BePDF / Xpdf          & marcicano                           \\
        Ekiga                 & flecha                              \\
        Empathy               & fefo                                \\
        Lib Folks (Telepathy) & kamiarc                             \\
        Scilab                & v1z, humannoise                     \\
        VxL                   & v1z                                 \\
        ImageMagick           & v1z                                 \\
        OpenOffice            & hick209                             \\
        Puredata              & v1z, automata, greenkobold, gilson, bzum \\
        Puredata OpenCV       & v1z                                 \\
        Puredata GEM          & v1z, fefo, hick209                  \\
        Puredata PDP          & v1z, fefo, hick209                  \\
        ChucK                 & rfabbri, automata                   \\
        ChucK MiniAudicle     & rfabbri, automata                   \\
        WebRTC                & automata                            \\
        OSC-Web               & automata                            \\
        Web-PD-GUI            & automata                            \\
        Live-Processing       & automata                            \\
        ChucK-Wiimote         & automata                            \\
        Audiolet              & automata                            \\
        Extempore             & automata                            \\
        \hline
    \end{tabular}
    \label{tabela:contribuicoes}
\end{table}

In one month, each developer contributed to many very large free software
projects. Many of the developers started the training with no knowledge of what
is free software and ended that period becoming a free software developer.

During that month, the same team developed the first version of the AA system
and used AA to manage their activities. Even while developing the system. All
the source code of AA --- the client that sends the logs and the AA Web server
--- is public available~\footnote{AA source code:
  \url{http://labmacambira.git.sourceforge.net/git/gitweb.cgi?p=labmacambira/aa}}
and all the AA sessions log of the whole team of ``Lab Macambira'' is also
on-line~\footnote{Logs of AA sessions:
  \url{http://labmacambira.git.sourceforge.net/git/gitweb.cgi?p=labmacambira/paainel}}.

After the training period, during more 6 months, the ``Macambiras'' worked on a
large range of free software projects, distributed on work groups --- each work
group focusing on a specific theme like video, audio and web --- and financed by
contracts and support of the ``Pont\~{a}o N\'{o}s Digitais''. In
Table~\ref{tabela:criados} we can see a list of the free software created by
``Lab Macambira'' since July of 2011.

\begin{table}
    \caption{Software projects created by ``Lab Macambira'' since July
      of 2011 with a short description and the technologies --- like
      programming languages or frameworks --- involved. It is
      interesting to note the heterogeneity of projects and its areas
      of application.}
    \small\begin{tabular}{|l|p{5cm}|l|}
        \hline
        Application & Description & Technologies involved \\ 
        \hline \hline
        AA            & Algorithmic Auto-regulation      & Python, PHP \\
        \hline
        \'{A}gora Communs & System for on-line deliberations & PHP \\
        \hline
        SIP           & Scilab Image Processing toolbox & C, Scilab \\
        \hline
        animal        & An Imaging Library              & C \\
        \hline
        TeDi          & Test Framework for Distance Transform
        Algorithms & C, Shellscript, Scilab \\
        \hline
        Macambot      & Multi-use IRC Bot               & Python \\
        \hline
        ``Confer\^{e}ncia Permanente'' & Platform for the permanent
        conference of the rights of minors & PHP, JavaScript \\
        \hline
        CPC           & Center for accounting of the Brazilian culture
        representation groups & Python, Django \\
        \hline
        Timeline      & Interactive time lines on the Web & JavaScript
        \\
        \hline
        Imagemap      & Interactive marking for on-line photos &
        JavaScript \\
        \hline
        ABT           & Program for real-time execution and musical
        rhythmic analysis & Python \\
        \hline
        EKP           & Emotional Kernel Panic & Python, ChucK \\
        \hline
        SOS           & Aggregation and diffusion of popular and native
        knowledge about health & Python, Django \\
        \hline
        Creative Economy & Platform for creative, collaborative and
        solidarity economy of the culture hubs and cultural entities &
        Python, Django \\
        \hline
        OpenID Integration & Adaptations to existing software for
        unified login through OpenID & PHP \\
        \hline
        pAAinel & Dashboard for the real-time visualization of Lab
        Macambira activity & Python, Django \\
        \hline
        Georef & Collection of scripts to be used as reference, which
        aims to be a GIS platform to map public data of use to
        citizens & Python, Django \\
        \hline
        AirHackTable & Software for an instrument which generates
        sound from flying origami tracked by webcams & Puredata,
        C/C++, Scilab \\
        \hline
        \end{tabular}
    \label{tabela:criados}
\end{table}

As of this writing the ``Lab Macambira'' have many software developers, and some
of the trained developers continue to work voluntarily in the project.

\section{Conclusions}
\label{conclusions}

%% brief introduction
In a scenario where GSD is growing as a popular form of software development,
not just on free software projects but in the whole software industry, we need
methodologies to deal with its disadvantages and at the same time to amplify its
advantages.

This paper has presented a methodology to GSD, being the development conducted
on large or small groups of software developers, working on different countries
or even at the same room. The AA methodology implements a simple system where
each developer take notes of his work generating a periodical log of small text
sentences. The sum of those sentences, along an entire session of work, results
in a complete report. The report is made public available through a Website and
be validated by other developers sorted aleatory by the AA Web server.

Instead of a merely work-management tool, AA act as a methodology to improve the
time sense of individuals, dividing their work on small sessions, and also
reducing the need of extensive reports or unnecessary meetings. By asking users
to write a minimal text sentence as a continuously log, AA does not disturb
developers concentrated on programming: developers just have to type some
characters, hit \textit{enter} and go back to coding.

AA application is not restricted to software development. As of this writing
there is a comic book studio~\footnote{Pula pirata: \url{http://pulapirata.com}}
starting to use AA to manage their activities. People with non-software
background, like social scientists, musicians and activists has also using AA
and contributing for its improvement.

For developer teams, we have experienced the use of AA to auto-regulate the work
of ``Lab Macambira'', a group of free software developers from Brazil. Since
July of 2011 the group have contributed and created new free and open source
software for a vast number of applications.

There are many aspects of the work which remain unfinished. New ways to report
logs --- the ``Twitter like'' messages --- from different interfaces like IRC,
Internet Messaging services and email can make the use of AA easy and
widespread, turning AA an ubiquitous system, presented on everyday communication
channels. Even the work logs generated since July of 2011 could be vastly
statistically analyzed aiming to recognize patterns in the behavior of
individuals and their productions.

We would like to conclude setting an important role of AA: being a free software
system and an open methodology, AA could be used to auto-manage groups of
individuals working on software or other kinds of activities. In this way, we
are interested to spread AA for those groups, to have even more developers
contributing in a collaborative way.

\section*{Acknowledgments}
%The authors would like to thank NSF and CNPq.

We would like to also thank AA: the present research and even this manuscript
was written using AA. The complete log is on-line at
\url{http://www.pulapirata.com/skills/aa}.


\nocite{last2003}
\nocite{german2003}
\nocite{carmel1999}
\nocite{carmel2001}
\nocite{komi2005}
\nocite{battin2001}

%{\small
%\input{paper-draft.bbl}
%%%%\bibliographystyle{splncs}
% this is a good style for drafting
%\bibliographystyle{abstract}
% this is a good style for finals
\bibliographystyle{acm}
\bibliography{aa.bib}
%}

\end{document}
