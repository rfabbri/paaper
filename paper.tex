%\documentclass[letterpaper,peerreview,12pt,compsoc,draftcls]{IEEEtran}
%\documentclass[10pt,twocolumn,letterpaper]{article}
\documentclass[letterpaper]{article}

%
% Final Checklist TODO XXX 
% - Spellcheck
% - check xxx's todo's etc
%
%\usepackage{ruler}
%\usepackage{cvpr}
%\usepackage{iccv}
\usepackage{times}
\usepackage{epsfig}
\usepackage{graphicx}
\usepackage{amsmath}
\usepackage{amssymb}
%\usepackage{amsthm}
\usepackage{mathtools,empheq}
\usepackage[tight,footnotesize]{subfigure}
%\usepackage{caption}
\usepackage{verbatim}
\usepackage{color}
\usepackage{nomencl}
\usepackage[nocompress]{cite}
\usepackage[pagebackref=true,breaklinks=true,letterpaper=true,colorlinks,bookmarks=false]{hyperref}
\usepackage{algorithm}
\usepackage{enumerate}
\usepackage{multirow}
\usepackage[width=122mm,left=12mm,paperwidth=146mm,height=193mm,top=12mm,paperheight=217mm]{geometry}


\newcommand{\intinf}{\int_{-\infty}^{\infty}}
\newcommand{\argmin}{\operatornamewithlimits{argmin}}

\newcommand{\grad}{\nabla}
\newcommand{\slant}{\sigma}
\newcommand{\R}{\mathbb{R}} % the reals
\newcommand{\skewm}[1]{{#1}_\times}

\renewcommand{\vec}[1]{\mathbf{#1}}

\newcommand{\lbar}{\overline}

\newcommand{\id}{\text{\emph{I}}}
\newcommand{\ransac}{\textsc{ransac}}
\newcommand{\sift}{\textsc{sift}}
\newcommand{\klt}{\textsc{klt}}
\newcommand{\svd}{\textsc{svd}}
\newcommand{\sfm}{\textsc{sfm}}


\newtheorem{thm}{Theorem}
\newtheorem{lem}[thm]{Lemma}

%\newtheorem{theorem}{Theorem}[section]
%\newtheorem{corollary}[theorem]{Corollary}
\newtheorem{corolary}[theorem]{Corollary}
%\newtheorem{proposition}[theorem]{Proposition}
%\newtheorem{lemma}[theorem]{Lemma}

%\theoremstyle{definition}
%\newtheorem{definition}{Definition}
%\newtheorem{remark}{Remark}[section]
\newtheorem{assumption}{Assumption}[section]
%\newtheorem{problem}{Problem}[section]
%\newtheorem{question}{Question}[section]
%\newtheorem{property}{Property}
\newtheorem{transformation}{Transformation}

\numberwithin{equation}{section}

\newcommand{\Gama}{\boldsymbol{\Gamma}}
\newcommand{\gama}{\boldsymbol{\gamma}}
\newcommand{\bsigma}{\boldsymbol{\sigma}}
%\newcommand{\Gama}{\Gamma}
%\newcommand{\gama}{\gamma}
\newcommand{\T}{\boldsymbol{T}}
\newcommand{\N}{\mathbf{N}}
\newcommand{\NSurface}{\mathbf{N}}
\newcommand{\Nlocal}{\overline{\N}} % normal in local coordinates
\newcommand{\balpha}{\boldsymbol{\alpha}}
\newcommand{\tDt}{t+\Delta t}
\newcommand{\bpsi}{\boldsymbol{\boldsymbol{\psi}}}
\newcommand{\bp}{\mathbf p}
\newcommand{\deldt}[1]{\frac{\partial#1}{\partial t}}
\newcommand{\ddt}[1]{\frac{d #1}{dt}}
\newcommand{\delds}[1]{\frac{\partial#1}{\partial s}}
\newcommand{\mybar}[1]{\overline{#1}}
\newcommand{\norm}[1]{\|#1\|}
\newcommand{\I}{\mathbf{I}}
\newcommand{\brho}{\boldsymbol{\rho}}
\newcommand{\lightrgb}{\boldsymbol{l}}
\newcommand{\B}{\boldsymbol{B}}
\renewcommand{\t}{\boldsymbol{t}}
\newcommand{\n}{\boldsymbol{n}}
\renewcommand{\b}{\boldsymbol{b}}
\newcommand{\e}{\boldsymbol{e}}
\newcommand{\f}{\boldsymbol{e}_3}
\newcommand{\ff}{\mathbf{f}}
\newcommand{\hf}{\boldsymbol{\hat{f}}}
\newcommand{\g}{\boldsymbol{g}}
\newcommand{\G}{\boldsymbol{G}}
\newcommand{\bc}{\boldsymbol{c}}
\newcommand{\Curve}{\Gamma}
%\newcommand{\X}{\boldsymbol{X}}
%\newcommand{\x}{\boldsymbol{x}}
\newcommand{\X}{\mathbf{X}}
\newcommand{\x}{\mathbf{x}}
\newcommand{\tilx}{\tilde x}
\newcommand{\tily}{\tilde y}
\newcommand{\tilgama}{\tilde \gama}
\newcommand{\ugama}{\hat{\gama}} %unit gama
\newcommand{\br}{\bar r}
\newcommand{\Kc}{\mathbf K_c}
\newcommand{\Kim} {\mathcal K_{im}}
\newcommand{\lepi}{\mathbf r}
\newcommand{\itan}{\tan^{-1}}
\newcommand{\uu}{\xi}
\newcommand{\buu}{\bar \uu}
\newcommand{\bvv}{\bar \vv}
\newcommand{\vv}{\eta}
\newcommand{\VV}{\mathbf{V}} % translational velocity
\newcommand{\VVspeed}{V} % translational velocity
\newcommand{\field}{\boldsymbol\chi}
\newcommand{\ufield}{\hat{\boldsymbol{\chi}}}
\newcommand{\fieldc}{\chi} % field component
\newcommand{\transl}{\mathcal{T}}
\newcommand{\rot}{\mathcal{R}}
\newcommand{\albedo}{\alpha}
\newcommand{\depth}{\rho}      % depth as z
\newcommand{\udepth}{{\hat{\rho}}} % depth along ray
\newcommand{\ttransl}{\T} % translation tangent
\newcommand{\surface}{\mathcal{M}} % surface/manifold
\newcommand{\surf}{\mathcal{M}} % surface/manifold short
\newcommand{\jacm}{\mathtt{J}} % Jacobian matrix
\newcommand{\xbar}{\bar x}
\newcommand{\ybar}{\bar y}
\newcommand{\zbar}{\bar z}

\newcommand{\bdelta}{\boldsymbol \delta}
%\newcommand{\X}{\boldsymbol{X}}
%\newcommand{\x}{\boldsymbol{x}}
%\newcommand{\X}{\mathbf{X}}
%\newcommand{\x}{\mathbf{x}}
\newcommand{\boldu}{\mathbf{u}}
\newcommand{\boldv}{\mathbf{v}}
\newcommand{\boldw}{\mathbf{w}}
% The following are not very good constructs it seems. Better to use just
% \begin{bmatrix}..\end{\bmatrix}
\newcommand{\datsqbr}[2][rrrrrrrrrrrrrrrrrrrrrrrrrrrrrrrrrrrr]{\left[
\begin{array}{#1}
#2\\
\end{array}
\right]
}
\newcommand{\tgtveloc}{\tilde\alpha} % real tangential velocity
\newcommand{\trace}{\text{trace}\,}
\newcommand{\benx}{x} % ben's aux. variable in da calibration paper

\newcommand{\ie}{{\it i.e.}}
\newcommand{\etc}{{\it etc}}
\newcommand{\eg}{{\it e.g.}}
\newcommand{\wrt}{{\it w.r.t. }}
\newcommand{\etal}{{\it et.~al.}}


\usepackage{url}
\usepackage{graphicx}
\usepackage[tight,footnotesize]{subfigure}
\usepackage{color}
\usepackage{verbatim}
\usepackage[pagebackref=true,breaklinks=true,letterpaper=true,colorlinks,bookmarks=false]{hyperref}

%%%%%%%%%%%%%%%%%%%%%%%%%%%%%%%%%%%%%%%%
% You have two versions of the macro
% \draftnote{My note}. The first version puts notes (e.g. My note in the example)
% into the margin of your document. The second formats the note as nothing. You
% 'comment out' the version of the macro you don't want (using a % at the
% beginning of the line).
%\newcommand{\draftnote}[1]{\marginpar{\tiny\raggedright\textsf{\hspace{0pt}#1}}}
\newcommand{\draftnote}[1]{\marginpar{\tiny\raggedright\textsf{\hspace{0pt}#1}}}
%\newcommand{\draftnote}[1]{}

% This one is just for the comments for in-line text.
\newcommand{\indraftnote}[1]{\textcolor{blue}{\texttt{\footnotesize[#1]}}}
\newcommand{\todo}[1]{\indraftnote{todo: #1}}
%\newcommand{\indraftnote}[1]{}


\begin{document}
\title{
    AA: the Algoritmic Autoregulation (Software Development) Methodology
}

\author{%
Renato~Fabbri \and Ricardo~Fabbri \and Vilson Vieira \and Alexandre Negrao \and Lucas Zambianchi
\and Marcos Mendonca \and Danilo Shiga
}

\maketitle
%\thispagestyle{empty}

\begin{abstract}
We present a new methodology for coordinating
teamwork called Algorithmic Autoregulation (AA). This methodology is based on recent social networking concepts and individual merit and exhibits convenient asynchronous, autodocumenting and autoregulating characteristics. Team
members take on an egalitarian and volutary role, by logging into
periodic ``AA sessions'' for an arbitrary duration (e.g.\ 2 hours per day).
During each session, a user creates a log composed of short text sentences
about their activity. These logs are
publicly aggregated in a Website and are peer-validated, as in code
review. A short screencast is ideally recorded at the end of each session to make
AA logs more understandable. This methodology seems to be well-suited for
increasing the efficiency of teams working on
Global Software Development (GSD). Recognized reasons for this are
the: 1) builtin asynchronous
on-demand communication, documentation and working hours comprobatives;
2) reduced need for central management,
meetings and time-consuming reports. Hence, AA
legitimizes and eases activities of a distributed software team.  It
enables groups to have new means to fund activities,
allowing for business models to emerge from
distributed software development. AA is proposed,
at it's core, as a way of having self-replicating hacker initiatives. These claims are
discussed in a real case-study of a distributed free software
hacker team called Lab Macambira.
\end{abstract}



\section{Introduction}

One of the defining features of modern times is the widening geographical
distribution of software teams~\cite{}. One example is the free software
movement. \todo{citar alguns exemplos}.

However, we have noticed how difficult it is to coordinate and fund free
software on a larger scale than currently available, when teams are very
heterogeneous containing not only volutaries and very experienced developers,
but also contractors from different backgrounds and cultures.

Another problem faced by modern software companies and other collectives are frequent
ineffective meetings, which are seldom focused to the interest to any
attendant. The result is that it has become the norm to participate in too many
meetings with the laptop open, which is very unproductive at the very least.
Coders like to code, to be productive, to have their hands-on their project to
do what they're best at. They hate to have to stop for too many meetings,
and they hate to have to write lengthy reports to justify their funding.
\todo{ler mais. cacm, etc}

AA is a methodology and an associated software system for coordinating
distributed teamwork.  Team members take on an egalitarian role, and stay
voluntarily logged in the system for part of their time (e.g., 2h/day), during
which they microblog periodical "tweets" about their activity. These microblog
logs are publicly aggregated and are validated by their own peers. Through AA,
we have a methodology and an associated system to validate and enable the
activities of a distributed software team. It implicitly legitimizes financing
the expansions of the team's activity. The AA methodology is specially useful
for coordinating distributed and decentralized teamwork, providing effective
means to asyncrhonously update different team members without the need for
syncrhonous unproductive meetings.

Figure~\ref{fig:mm} summarizes our methodology.


\begin{figure}
\begin{center}
   \includegraphics[width=0.8\linewidth,keepaspectratio=true]{figs/aa-mm.png}
\end{center}
   \caption{
   Mindmap of our methodology
   }
\label{fig:mm}
\end{figure}

\todo{alerts work for greater consciousness of time}

\section{Related Work}
\todo{survey other methodologies such as agile etc}
\todo{probably a good source http://agilemanifesto.org/}

\section{The AA Methodology}

\section{AA Session}

From the developer perspective, the AA methodology is based on creating pretty
small high perspective reports of what they're doing in a specific timeframe,
that can be something between 5 to 15 minutes, depending in what is more
confortable for the developer, an AA Session would be a period of at least 2hrs
doing these reports. The developer can set reminders to show up when its time
to make a new report.

The objective of the flexible timeframe and reminders is to don't generate much
overhead for the developer to be in an AA session, so he can make the reports
and also concentrate in his code.

Each report can be sent directly to the an online server, or cached locally in
a log for sending later, this is necessary to make possible for developers that are
offline to use AA without problems.

The developer can also add a screencast in the end of session to make a summary
of what has been done in the session, explaining with his words and showing his
results, possibly making some points more easily understandable then with only
the reports from the AA session.



\section{The AA Methodology}

\section{Conclusion}

%\section*{Acknowledgments}
%The authors would like to thank NSF and CNPq.

%{\small
%\input{paper-draft.bbl}
%%%%\bibliographystyle{splncs}
% this is a good style for drafting
%\bibliographystyle{abstract}
% this is a good style for finals
\bibliographystyle{acm}
\bibliography{personal}
%}

\end{document}
