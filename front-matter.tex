\title{
AA Distributed Software Development Methodology
}

\author{%
Renato~Fabbri \and Ricardo~Fabbri \and Vilson Vieira \and Alexandre Negr\~{a}o \and Lucas Zambianchi
\and Marcos Mendon\c{c}a \and Daniel Penalva \and Danilo Shiga\\[1em]
\small{
Instituto de F\'{i}sica de S\~{a}o Carlos (IFSC), Universidade de
S\~{a}o Paulo (USP), Brazil}\\[0.5em]
\small{Instituto Polit\'{e}cnico, Universidade do Estado do Rio de
Janeiro, Nova Friburgo, RJ, Brazil}\\[0.5em]
\url{LabMacambira.sourceforge.net}.
}


\maketitle
%\thispagestyle{empty}

\begin{abstract}
We present a new self-regulating methodology for coordinating distributed team
work called Algorithmic Auto-regulation (AA), based on recent social networking
concepts and individual merit. Team members take on an egalitarian role, and
stay voluntarily logged into so-called AA sessions for part of their time (e.g.\
2 hours per day), during which they create periodical logs --- short text
sentences --- they wish to share about their activity with the team. These logs
are publicly aggregated in a Website and are peer-validated after the end of a
session, as in code review.  A short screencast is ideally recorded at the end
of each session to make AA logs more understandable.  This methodology has shown
to be well-suited for increasing the efficiency of distributed teams working on
what is called Global Software Development (GSD), as observed in our experience
in actual real-world situations.  This efficiency boost is mainly achieved through 1)
built-in asyncrhonous on-demand communication, documentation of work products
and processes, and 2) reduced need for central management, meetings or
time-consuming reports. Hence, the AA methodology legitimizes and facilitates
the activities of a distributed software team.  It thus enables other entities
to have a solid means to fund these activities, allowing for new and concrete
business models to emerge for very distributed software development. AA has been
proposed, at its core, as a way of sustaining self-replicating hacker
initiatives. These claims are discussed in a real case-study of running a
distributed free software hacker team called Lab Macambira.  
\end{abstract}
