\title{
    AA: the Algoritmic Autoregulation (Software Development) Methodology
}

\author{%
Renato~Fabbri \and Ricardo~Fabbri \and Vilson Vieira \and Alexandre Negrao \and Lucas Zambianchi
\and Marcos Mendonca \and Danilo Shiga
}

\maketitle
%\thispagestyle{empty}

\begin{abstract}
We present a new methodology for coordinating
teamwork called Algorithmic Autoregulation (AA). This methodology is based on recent social networking concepts and individual merit and exhibits convenient asynchronous, autodocumenting and autoregulating characteristics. Team
members take on an egalitarian and volutary role, by logging into
periodic ``AA sessions'' for an arbitrary duration (e.g.\ 2 hours per day).
During each session, a user creates a log composed of short text sentences
about their activity. These logs are
publicly aggregated in a Website and are peer-validated, as in code
review. A short screencast is ideally recorded at the end of each session to make
AA logs more understandable. This methodology seems to be well-suited for
increasing the efficiency of teams working on
Global Software Development (GSD). Recognized reasons for this are
the: 1) builtin asynchronous
on-demand communication, documentation and working hours comprobatives;
2) reduced need for central management,
meetings and time-consuming reports. Hence, AA
legitimizes and eases activities of a distributed software team.  It
enables groups to have new means to fund activities,
allowing for business models to emerge from
distributed software development. AA is proposed,
at it's core, as a way of having self-replicating hacker initiatives. These claims are
discussed in a real case-study of a distributed free software
hacker team called Lab Macambira.
\end{abstract}

