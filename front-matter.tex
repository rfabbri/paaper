\title{
AA Distributed Software Development Methodology
}

\author{%
Renato~Fabbri \and Ricardo~Fabbri \and Vilson Vieira \and Alexandre Negrao \and Lucas Zambianchi
\and Marcos Mendonca \and Danilo Shiga
}

\maketitle
%\thispagestyle{empty}

\begin{abstract}
We present a new self-regulating methodology for coordinating
distributed team work called Algorithmic Auto-regulation (AA), based
on recent social networking concepts and individual merit. Team
members take on an egalitarian role, and stay voluntarily logged into
so-called AA sessions for part of their time (e.g.\ 2 hours per day),
during which they create periodical logs --- short text sentences ---
they wish to share about their activity with the team. These logs are
publicly aggregated in a Website and are peer-validated, as in code
review. A video is generally recorded by the members to make their
work logs more understandable. This methodology is well-suited for
increasing the efficiency of distributed teams working on what is
called Global Software Development (GSD) through asynchronous
on-demand communication, reducing the need for central management,
unproductive meetings or time-consuming reports. The AA methodology
also legitimizes the activities of a distributed software team.  It
thus enables entities to have a solid means to fund these activities,
allowing for new and concrete business models to emerge for very
distributed software development. AA as a methodology is also at the
core of having self-replicating hacker initiatives. These claims are
discussed in a real case-study of running a distributed free software
hacker team called Lab Macambira.
\end{abstract}

