\title{%
AA Distributed Software Development Methodology
}

\author{%
Renato~Fabbri \and Ricardo~Fabbri \and Vilson Vieira \and Alexandre Negrao \and Lucas Zambianchi
\and Marcos Mendonca \and Danilo Shiga
}

\maketitle
%\thispagestyle{empty}


\begin{abstract}
We present a new self-regulating methodology for coordinating
distributed teamwork called Algorithmic Autoregulation (AA),  based on recent social
networking concepts and individual merit. Team members take on an egalitarian role, and stay
voluntarily logged into so-called AA sessions for part of their time (e.g.,
2h/day), during which they check for open tickets and microblog periodical
"tweets" or status messages they wish to share about their activity with the
team. These status logs are publicly aggregated and are peer-validated,
as in code review, and a public videolog is recorded summarizing daily experiences.
This methodology is well-suited for increasing the efficiency of distributed
teams through asynchronous on-demand communication, reducing the need for
central management, unproductive meetings or time-consuming reports.  The AA
methodology also legitimizes the activities of a distributed software team.  It
thus enables entities to have a solid means to fund these activities, allowing
for new and concrete business models to emerge for very distributed software
development. AA as a methodology is also at the core of having self-replicating
hacker initiatives. These claims are discussed in a real case-study of running a distributed
free software hacker team called Lab Macambira.
\end{abstract}

